%!TEX TS-program = xelatex
\documentclass[]{friggeri-cv}
\usepackage{afterpage}
\usepackage{hyperref}
\usepackage[spanish]{babel}
\usepackage{color}
\usepackage{xcolor}
\hypersetup{%
    pdftitle={Curriculum Vitae},
    pdfauthor={Arturo Volpe},
    pdfsubject={Resumen de vida Profesional},
    pdfkeywords={},
    colorlinks=false,       % no link border color
   allbordercolors=white    % white border color for all
}
\addbibresource{bibliography.bib}
\RequirePackage{xcolor}
\definecolor{pblue}{HTML}{0395DE}

\begin{document}
\header{Arturo}{Volpe}
{Ingeniero en Informática}

% Fake text to add separator
\fcolorbox{white}{gray}{\parbox{\dimexpr\textwidth-2\fboxsep-2\fboxrule}{%
.....
}}

% In the aside, each new line forces a line break
\begin{aside}
    \section{Celular}
        (0972) 105 - 788
        ~
    \section{Correo}
        \href{mailto:arturovolpe@gmail.com}{arturovolpe@gmail.com}
        ~
    \section{Repositorios}
        \href{https://github.com/avolpe}{github.com/avolpe}
        ~
    \section{Blog}
        \href{https://avolpe.github.io}{avolpe.github}
        ~
\section{Idiomas}
\textbf{Español}\includegraphics[scale=0.40]{img/5stars.png}
\textbf{Ingles}\includegraphics[scale=0.40]{img/3stars.png}
\end{aside}

\section{Experiencia}
\begin{entrylist}
\entry
    {2015-}
    {Centro de Desarrollo Sostenible S.A.}
    {Ingeniero de Software}
    {\proyecto
        {\href{http://bacheando.com/}{Bacheando}}
        {2015}
        {Programador}
        {Proyecto de geolocalización de baches, perdidas, hundimientos y
            cortes de la municipalidad de Asunción.
            Posee un componente web, así como una aplicación móvil para
            dispositivos Android.}
        {\begin{itemize}
            \item \textbf{JavaScript}: angular, d3, bootstrap, leaflet.
            \item \textbf{Java}: JavaEE, Jax-Rs, Hibernate, AndroidSDK.
        \end{itemize}} 
    \proyecto
        {Portal de datos abiertos para el Poder Judicial}
        {2015}
        {Full stack}
        {Portal de datos abiertos para el Poder Judicial, incluyendo varias
        	fuentes de datos.}
        {\begin{itemize}
            \item \textbf{Front-end}: angular, bootstrap-ui, underscore, jQuery datatables.
            \item \textbf{Back-end}: JavaEE, Jax-Rs, PostgreSQL, JPA.
            \item \textbf{Infraestructura}: Docker, Nginx, Wildfly, Redis, PostgreSQL, CUPS.
        \end{itemize}} 
    \proyecto
        {Sistema Integral de Atención y Historia clínica para el Sanatorio Christian S.R.L.}
        {2015-}
        {Arquitecto y Full stack}
        {Portal de datos abiertos para el Poder Judicial, incluyendo varias
        	fuentes de datos.}
        {\begin{itemize}
            \item \textbf{JavaScript}: angular, bootstrap, leaflet.
            \item \textbf{Java}: JavaEE, Jax-Rs, Wildfly 10.
        \end{itemize}} }

\entry
    {2012 - 2015}
    {Facultad Politécnica}
    {Arquitecto de Software}
    {Ingrese como programador junior, desempeñando cargos como la creación de
        aplicaciones base para el desarrollo, tareas de diseño de interfaces
        como de lógica de negocio, integración de componentes, e investigación
        de tecnológicas.

        Proyectos:
        \begin{itemize}
            \item Sistema Integral de Gestión Hospitalaria (Hospital de
                Clínicas)
            \item Sistema de Gestión Convergente de Nueva Generación (Copaco)
        \end{itemize}}
\entry
    {2011 - 2012}
    {Konecta S.A.}
    {Programador Junior}
    {Como programador junior, mis responsabilidades incluían el diseño y
        desarrollo de software para dispositivos móviles, y la creación de
        servicios webs, tanto servicios que interactuaban directamente con
        aplicaciones móviles, como otros que interactuaban con aplicaciones webs
        \\}
\entry
    {2011 - 2011}
    {Global S.A.}
    {Programador Junior}
    {Como programador junior mis funciones iban desde correcciones y pruebas de
        aplicaciones en fase final de desarrollo, hasta realizar funciones
        relacionadas con el mantenimiento de la base de datos y su correcto
        funcionamiento. Además de administración de servidores. \\}
\entry
    {2010 - 2011}
    {Bunker Games}
    {Programador}
    {Como programador principal y único, mis responsabilidades iban desde la
        correcto funcionamiento del videojuego, la integración del entorno
        gráfico, programación de la inteligencia artificial, cálculo de
        recorridos, optimización, etc. \\}
\entry
    {2005 - 2008}
    {Cyber Bunker}
    {Técnico de computadoras}
    {Como técnico mi responsabilidad era el correcto funcionamiento de las
        maquinas pertenecientes al cyber, así como la configuración de la red.
        \\}
\entry
    {2006 - 2007}
    {Colegio de San José}
    {Pasante}
    {Ingrese como programador junior, desempeñando cargos como la
creación de aplicaciones base para el desarrollo, tareas de diseño de
interfaces como de lógica de negocio, integración de componentes, e
investigación de tecnológicas. \\}

\end{entrylist}


\newpage
\section{Experiencia Independiente}
\proyectof
    {2015}
    {\href{https://apps.facebook.com/pechugon_pepe}{PepeChugon Game}}
    {\href{http://www.animata.com.py/}{Animata}}
    {Consiste en un juego multiplataforma~(Android, Facebook Canvas y Windows)
        y un servidor donde se almacena la información y el rendimiento de los
        usuarios. El juego fue desarrollado para la campaña publicitaria del día
        del niño.}
    {\begin{itemize}
            \item \textbf{Juego}: Unity3D, Facebook Unity Plugin.
            \item \textbf{Backend}: Java EE, OpenShift, PostgreSQL, Facebook
                JavaScript API, Primefaces, Primefaces Mobile.
        \end{itemize}}

    \section{Experiencia académica}
\clase
    {2016}
    {Curso de Verano}
    {\href{http://www.pol.una.py/cursosverano/index.php?option=com_content&view=article&layout=edit&id=91}{Innovación
            social y transparencia mediante datos abiertos}} 
    {Auxiliar}
    {FP-UNA}
    {El curso busca crear una comunidad de personas que puedan aportar ideas
        innovativas con alto impacto social y que fomenten la transparencia en
        el uso de los recursos públicos. }
\clase
    {2016-}
    {Licenciatura en ciencias Informáticas}
    {Algoritmos I} 
    {Auxiliar}
    {FP-UNA}
    {Introducción a algoritmos, conceptos básicos y programación con SLE2.}

\clase
    {2016-}
    {Ingeniería en informática}
    {Gestión de Centro de Cómputos} 
    {Auxiliar}
    {FP-UNA}
    {}

\clase
    {2016}
    {}
    {Introducción a JAVA} 
    {}
    {MSPBS}
    {Introducción a Java para.}


\section{Educación}
\begin{entrylist}
  \entry
    {2008 - 2015}
    {Ingeniero en Informática}
    {Facultad Politécnica - Universidad Nacional de Asunción}
    {}
  \entry
    {2005 - 2007}
    {Bachiller Técnico en Informática}
    {Colegio de San José}
    {}
\end{entrylist}



\section{Referencias}
\begin{entrylist}
    \entry{M.Sc.}{Martin Abente}{}{}
    \entry{M.Sc.}{Guillermo Gonzalez}{}{}
    \entry{M.Sc.}{Maria Elena Garcia}{}{}
\end{entrylist}

\begin{flushleft}
    \emph{\today{}}
\end{flushleft}
\begin{flushright}
    \emph{Arturo Volpe}
\end{flushright}


\end{document}
