%!TEX TS-program = xelatex
\documentclass[]{friggeri-cv}
\usepackage{afterpage}
\usepackage{hyperref}
\usepackage{csquotes}
\usepackage[spanish]{babel}
\usepackage{color}
\usepackage{xcolor}
\hypersetup{%
    pdftitle={Curriculum Vitae},
    pdfauthor={Arturo Volpe},
    pdfsubject={Resumen de vida Profesional},
    pdfkeywords={},
    colorlinks=false,       % no link border color
   allbordercolors=white    % white border color for all
}
\addbibresource{bibliography.bib}
\RequirePackage{xcolor}
\definecolor{pblue}{HTML}{0395DE}

\begin{document}
\header{Arturo}{Volpe}
{Ingeniero en Informática}

% Fake text to add separator
\fcolorbox{gray}{gray}{\parbox{\dimexpr\textwidth-40\fboxsep-2\fboxrule}{%
.....
}}

% In the aside, each new line forces a line break
\begin{aside}
	\section{}~
    \section{}~
    \section{}~
    \section{Celular}
        (0972) 105 - 788
        ~
    \section{Correo}
        \href{mailto:arturo@volpe.com.py}{arturo@volpe.com.py}
        \href{mailto:arturovolpe@gmail.com}{arturovolpe@gmail.com}
        ~
    \section{Repositorios}
        \href{https://github.com/avolpe}{github.com/avolpe}
        ~
    \section{Blog}
        \href{https://www.volpe.com.py}{volpe.com.py}
        ~
\section{Idiomas}
\textbf{Español}\includegraphics[scale=0.40]{img/5stars.png}
\textbf{Ingles}\includegraphics[scale=0.40]{img/4stars.png}
\end{aside}


\section{Experiencia}
\begin{entrylist}
\entry
    {2015-2017}
    {\href{http://www.cds.com.py}{Centro de Desarrollo Sostenible S.A.}}
    {Arquitecto de Software}
    {\proyecto
        {Financiera el Comercio}
        {2016-2017}
        {Arquitecto de Software y Gestión de Infraestructura}
        {}
        {\begin{itemize}
            \item \textbf{Arquitectura de sistemas}: Sistemas satélite de la Financiera, incluyendo BancaWeb, Aplicaciones móviles, sistemas de gestión internos e integración con sistemas Externos.
            \item \textbf{Integración continua}: Automatización del proceso de construcción de todos los sistemas externos y creación de estructura para automatización de pruebas de integración.
            \item \textbf{Producción}: Monitoreo de sistemas.
        \end{itemize}} 
        {976}
    
    \proyecto
        {\href{https://www.mytappweb.com}{myTapp}}
        {2016-2017}
        {Arquitecto, desarrollo back-end y front-end}
        {Plataforma para gestión de comercios y pagos integrados.}
        {\begin{itemize}
            \item \textbf{Móvil}: ionic.
            \item \textbf{Back-end}: JavaEE, Jax-Rs, PostgreSQL, JPA, EJB.
            \item \textbf{Infraestructura}: Docker, Nginx, Wildfly, Redis, PostgreSQL, AWS, Jenkins, Sonar.
        \end{itemize}} 
        {1920}
        
    \proyecto
        {\href{http://www.cds.com.py/productos/kinematics/}{Kinematic}}
        {2016-2017}
        {Arquitecto de Software y desarrollo}
        {Plataforma automatizada de entrenamiento de tiro de balón.}
        {\begin{itemize}
            \item \textbf{Core}: python, OpenCV, c++.
            \item \textbf{Front-end}: python, django, html5.
        \end{itemize}}
        {300}
        
    \proyecto
        {\href{https://mawio.net/}{MAWIO}}
        {2016-2017}
        {Arquitecto de Software y desarrollo}
        {Mapa Web Interactivo de Obras de la Essap}
        {\begin{itemize}
            \item \textbf{Front-end}: angular, bootstrap-ui, jQuery datatables.
            \item \textbf{Back-end}: JavaEE, Jax-Rs, PostgreSQL, JPA, EJB.
            \item \textbf{Infraestructura}: Docker, Nginx, Wildfly, Redis, PostgreSQL.
        \end{itemize}}
        {300}
        
    \proyecto
        {\href{http://www.cds.com.py/productos/sistemas-de-gestion/sistema-de-gestion-hospitalaria/}{SGH}}
        {2015-2017}
        {Arquitecto y líder de proyecto}
        {Sistema de Gestión Hospitalaria, que incluye la atención 
        	al cliente, manejo de personal, inventario, historia clínica, etc.}
        {\begin{itemize}
            \item \textbf{Front-end}: angular, bootstrap-ui, jQuery datatables.
            \item \textbf{Back-end}: JavaEE, Jax-Rs, PostgreSQL, JPA, EJB.
            \item \textbf{Infraestructura}: Docker, Nginx, Wildfly, Redis, PostgreSQL, CUPS.
        \end{itemize}}
        {1920}
      }
      {}
    
  \entry
    {2015-2017}
    {\href{http://www.cds.com.py}{Centro de Desarrollo Sostenible S.A.}}
    {Arquitecto de Software}
    {
        
    \proyecto
        {Portal de datos abiertos para el Poder Judicial}
        {2015}
        {Full stack}
        {Portal de datos abiertos para el Poder Judicial, incluyendo varias
        	fuentes de datos.}
        {\begin{itemize}
            \item \textbf{Front-end}: angular, bootstrap-ui, jQuery datatables.
            \item \textbf{Back-end}: Wildfly, Jax-Rs, Microsoft SQL Server, JPA.
        \end{itemize}}
        {352} 
    \proyecto
        {\href{http://bacheando.com/}{Bacheando}}
        {2015}
        {Programador}
        {Proyecto de geolocalización de baches, perdidas, hundimientos y
            cortes de la municipalidad de Asunción.
            Posee un componente web, así como una aplicación móvil para
            dispositivos Android.}
        {\begin{itemize}
            \item \textbf{JavaScript}: angular, d3, bootstrap, leaflet.
            \item \textbf{Java}: Wildfly, Jax-Rs, Hibernate, AndroidSDK.
        \end{itemize}}
        {176} 
        
	}
    {}


\entry
    {2012 - 2015}
    {Facultad Politécnica}
    {Arquitecto de Software}
    {Ingrese como programador junior, desempeñando cargos como la creación de
        aplicaciones base para el desarrollo, tareas de diseño de interfaces
        como de lógica de negocio, integración de componentes, e investigación
        de tecnológicas.

        Proyectos:
        \begin{itemize}
            \item Sistema Integral de Gestión Hospitalaria (Hospital de
                Clínicas)
            \item Sistema de Gestión Convergente de Nueva Generación (Copaco)
        \end{itemize}}
    {5280}
\entry
    {2011 - 2012}
    {Konecta S.A.}
    {Programador Junior}
    {Como programador junior, mis responsabilidades incluían el diseño y
        desarrollo de software para dispositivos móviles, y la creación de
        servicios webs, tanto servicios que interactuaban directamente con
        aplicaciones móviles, como otros que interactuaban con portales webs
        \\}
    {1408}
\entry
    {2011 - 2011}
    {Global S.A.}
    {Programador Junior}
    {Como programador junior mis funciones iban desde correcciones y pruebas de
        aplicaciones en fase final de desarrollo, hasta realizar funciones
        relacionadas con el mantenimiento de la base de datos y su correcto
        funcionamiento. Además de administración de servidores. \\}
    {300}
\entry
    {2010 - 2011}
    {Bunker Games}
    {Programador}
    {Como programador principal y único, mis responsabilidades iban desde la
        correcto funcionamiento del videojuego, la integración del entorno
        gráfico, programación de la inteligencia artificial, cálculo de
        recorridos, optimización, etc. \\}
    {1408}
\entry
    {2006 - 2007}
    {Colegio de San José}
    {Pasante}
    {Ingrese como programador junior, desempeñando cargos como la
creación de aplicaciones base para el desarrollo, tareas de diseño de
interfaces como de lógica de negocio, integración de componentes, e
investigación de tecnológicas. \\}
    {300}

\end{entrylist}

\newpage
\section{Experiencia Independiente}
\proyectof
    {2015}
    {\href{https://apps.facebook.com/pechugon_pepe}{PepeChugon Game}}
    {\href{http://www.animata.com.py/}{Animata}}
    {Consiste en un juego multiplataforma~(Android, Facebook Canvas y Windows)
        y un servidor donde se almacena la información y el rendimiento de los
        usuarios. El juego fue desarrollado para la campaña publicitaria del día
        del niño.}
    {\begin{itemize}
            \item \textbf{Juego}: Unity3D, Facebook Unity Plugin.
            \item \textbf{Backend}: Java EE, OpenShift, PostgreSQL, Facebook
                JavaScript API, Primefaces, Primefaces Mobile.
        \end{itemize}}
    {300}
        

\section{Experiencia académica}
\clase
    {2016}
    {Curso de Verano}
    {\href{http://www.pol.una.py/cursosverano/index.php?option=com_content&view=article&layout=edit&id=91}{Innovación
            social y transparencia mediante datos abiertos}} 
    {Auxiliar}
    {FP-UNA}
    {El curso busca crear una comunidad de personas que puedan aportar ideas
        innovativas con alto impacto social y que fomenten la transparencia en
        el uso de los recursos públicos. }
\clase
    {2016}
    {Licenciatura en ciencias Informáticas}
    {Algoritmos I y II} 
    {Auxiliar de enseñanza}
    {FP-UNA}
    {Introducción a algoritmos, conceptos básicos y programación con SLE2 y C.}

\clase
    {2016}
    {Ingeniería en informática}
    {Gestión de Centro de Cómputos} 
    {Auxiliar de enseñanza}
    {FP-UNA}
    {}

\section{Educación}
\begin{entrylist}
  \educacion
    {2008 - 2015}
    {Ingeniero en Informática}
    {Facultad Politécnica - Universidad Nacional de Asunción}
    {\textbf{Tesis de grado:} Juegos serios como apoyo a la formación de profesionales: Una aplicación al área de enfermería}
  \educacion
    {2005 - 2007}
    {Bachiller Técnico en Informática}
    {Colegio de San José}
    {}
\end{entrylist}

\begin{absolutelynopagebreak}
\section{Referencias}
\begin{entrylist}
    \educacion{M.Sc.}{Martin Abente}{}{}
    \educacion{M.Sc.}{Guillermo Gonzalez}{}{}
    \educacion{M.Sc.}{Maria Elena Garcia}{}{}
\end{entrylist}
\end{absolutelynopagebreak}

\begin{flushleft}
    \emph{\today{}}
\end{flushleft}
\begin{flushright}
    \emph{Arturo Volpe}
\end{flushright}



\end{document}
