%!TEX TS-program = xelatex
\documentclass[]{friggeri-cv}
\usepackage{afterpage}
\usepackage{hyperref}
\usepackage{csquotes}
\usepackage[spanish]{babel}
\usepackage{color}
\usepackage{xcolor}
\hypersetup{%
    pdftitle={Curriculum Vitae},
    pdfauthor={Arturo Volpe},
    pdfsubject={Resumen de vida Profesional},
    pdfkeywords={},
    colorlinks=false,       % no link border color
   allbordercolors=white    % white border color for all
}
\addbibresource{bibliography.bib}
\RequirePackage{xcolor}
\definecolor{pblue}{HTML}{0395DE}

\begin{document}
\header{Arturo }{Volpe}
{Arquitecto de Software}

% Fake text to add separator
\fcolorbox{gray}{gray}{\parbox{\dimexpr\textwidth-40\fboxsep-2\fboxrule}{%
.....
}}

% In the aside, each new line forces a line break
\begin{aside}
	\section{}~
    \section{}~
    \section{}~
    \section{Celular}
        +595972105788
        ~
    \section{Correo}
        \href{mailto:arturo@volpe.com.py}{arturo@volpe.com.py}
        \href{mailto:arturovolpe@gmail.com}{arturovolpe@gmail.com}
        ~
    \section{Código}
        \href{https://github.com/avolpe}{github.com/avolpe}
        \href{https://gitlab.com/avolpe}{gitlab.com/avolpe}
        ~
    \section{Blog}
        \href{https://www.volpe.com.py}{www.volpe.com.py}
        ~
\section{Idiomas}
\textbf{Español}\includegraphics[scale=0.40]{img/5stars.png}
\textbf{Inglés}\includegraphics[scale=0.40]{img/3stars.png}
\end{aside}


\section{Experiencia Profesional}
\begin{entrylist}
\entry
    {2018 - act.}
    {\href{https://fintech.works}{fintech.works}}
    {}
    { \proyecto
        {PRISMA}
        {2022-2023}
        {}
        {Arquitecto de Software}
        {
            PRISMA es una organización que brinda servicios de medios de pago en diversas verticales. 
            A modo de ejemplo: Adquirencia Visa/Mastercard, red de POS, Emisión y Procesamiento de 
            medios de pagos con alcance nacional e internacional, prevención de fraudes, etc. 

            Implementación de un sistema de procesamiento multiemisor basado en tecnología jPos y 
            desplegado en la nube utilizando kuberetes.
        } 
        {170}
 
    \proyecto
        {BEPSA}
        {2020-2022}
        {}
        {Arquitecto de Software}
        {
            BEPSA es una empresa enfocada en pagos digitales con servicios en diferentes áreas. A modo 
            de ejemplo: red de POS, red de ATM, procesador Emisor y procesador Adquirente.

            Como arquitecto del proceso de reingenieria del sistema de acquiriencia fuí el encargado
            de diseñar una plataforma lista para ser desplegada en la nube de alto rendimiento basada
            en kubernetes y jPos.
        } 
        {170}

    \proyecto
        {Wally}
        {2018-2020}
        {}
        {Arquitecto de Software}
        {
            NETEL Paraguay es una EMPE (Entidad de Medio de Pago Electrónico) que opera en Paraguay 
            a través de la marca Wally, una billetera móvil disponible para dispositivos Android y iOS

            Wally es una billetera electrónica, como arquitecto me encargue de realizar
            la integración con sistemas de procesamiento de tarjetas, diseñar los esquemas de 
            despliegue, y del diseño de la arquitectura de la aplicación y middleware.} 
        {170}

    }
    {xxxx}
\entry
    {2015-2017}
    {\href{https://www.cds.com.py}{Centro de Desarrollo Sostenible S.A.}}
    {}
    {\proyecto
        {Financiera el Comercio}
        {2016-2017}
        {}
        {Arquitecto de Software}
        {
        \textbf{Arquitectura y mantenimiento} de sistemas satélite de la Financiera, incluyendo BancaWeb, aplicaciones móviles, sistemas de gestión internos e integración con sistemas Externos. Encargado de proceso de 
        \textbf{Integración continua} a través de la automatización del proceso de construcción de todos los sistemas y automatización de pruebas de integración y e2e. En el área de \textbf{producción}, automatización del proceso de monitoreo de sistemas y generación de alertas.
        } 
        {976}
    
    \proyecto
        {\href{https://www.mytappweb.com}{myTapp}}
        {2016-2017}
        {Plataforma para gestión de comercios y pedidos con integración a plataformas de pago.}
        {Arquitecto de Software}
        {
        Desarrollo de aplicación móvil implementada con  \textbf{ionic}, 
        backend utilizando el stack JavaEE~(jax-rs, JPA, EJB) y una base de datos
        PostgreSQL.  Además, implementé un esquema de \textit{CD} para este proyecto,
        utilizando \textit{Jenkins, Rancher, Docker, Arquillian, SonarQUBE, AWS y Scaleway}.  
        En producción se utilizaron  otros sistemas complementarios de código abierto 
        como son: \textit{PictShare, pictfit} para la gestión de imágenes, redis para una
        caché de alto rendimiento y nginx como proxy reverso.  El sistema en producción 
        funciona sobre AWS (RDS, y instancias EC2 manejadas a 
        través de un Rancher). Además incluye un sistema de monitoreo del sistema
        utilizando \textit{Graphite y Dropwizard}.
        } 
        {}
        
    \proyecto
        {\href{http://www.cds.com.py/productos/kinematics/}{Kinematic}}
        {2016-2017}
        {Plataforma automatizada de entrenamiento de tiro de balón}
        {Arquitecto de Software}
        {Consta de dos componentes principales, un \textbf{core} de detección de movimiento de pelotas en movimiento desarrollado en python y c++ utilizando OpenCV. Así mismo, un componente de \textbf{front-end}, el cual se encuentra desarrollado con python utilizando django.}
        {300}
        
    \proyecto
        {\href{http://www.farmacenter.com.py/}{Farmacenter}}
        {2016-2017}
        {Plataforma de comercio electronico}
        {Arquitecto de Software}
        {
            Optimización y mejoras en portal de e-commerce de la cadena de farmacias más grande de Paraguay. 
            Como parte del proyecto me encargue de la optimización de la plataforma desarrollada en 
            \textbf{PHP} y \textbf{MySQL}.
        }
        {300}
        
    \proyecto
        {\href{https://mawio.net/}{MAWIO}}
        {2016-2017}
        {Mapa Web Interactivo de Obras de la Essap}
        {Arquitecto y full stack}
        {SPA desarrollada utilizando principalmente en el front-end \textbf{angular, leaflet y jQuery datatables}, en el back-end \textbf{Jaxa EE (jax-rs, JPA y EJB),}, una base de datos \textbf{PostgreSQL con PostGIS}, y manejo del entorno de desarrollo y producción utilizando \textbf{docker, nginx, wildfly 9 y redis}}
        {300}
        
      }
      {}
   
  \entry
    {2015-2017}
    {\href{http://www.cds.com.py}{Centro de Desarrollo Sostenible S.A.}}
    {}
    {
        
    \proyecto
        {\href{http://www.cds.com.py/productos/sistemas-de-gestion/sistema-de-gestion-hospitalaria/}{Sistema de Gestión Hospitalaria}}
        {2015-2017}
        {Sistema integral para hospitales}
        {Arquitecto y líder de proyecto}
        {SPA desarrollada utilizando principalmente en el front-end 
        \textbf{angular, boostrap-ui y jQuery datatables}, en el back-end 
        \textbf{Jaxa EE (jax-rs, JPA y EJB)}, una base de datos PostgreSQL, 
        y manejo del entorno de desarrollo y producción utilizando 
        \textbf{docker, nginx, wildfly 9 y redis}}
        {1920}
        
    \proyecto
        {Portal de datos abiertos para el Poder Judicial}
        {2015}
        {Portal de datos abiertos que incluye varias fuentes de datos.}
        {full stack}
        {Portal de datos abiertos desarrollado como una SPA utilizando 
        \textbf{angular, boostrap-ui y jQuery datatables}, en el back end 
        se utilizó el stack estándar de Java EE (jax-rs, JPA, EJB), como base 
        de datos se utilizó Microsoft SQL Server 2008.}
        {352} 
        
    \proyecto
        {\href{http://bacheando.com/}{Bacheando}}
        {2015}
        {Proyecto de localización de incidentes urbanos}
        {full stack}
        {Permite la geolocalización de baches, pérdidas, hundimientos y
            cortes de la municipalidad de Asunción.
            Posee un componente web (desarrollado utilizando 
            \textbf{angular, d3, boostrap y leaflet}), así como una 
            aplicación móvil para dispositivos Android y un servidor desarrollado 
            utilizando wildfly y Java EE(jax-rs, JPA).}
        {176} 
        
	}
    {}


\entry
    {2012 - 2015}
    {Facultad Politécnica}
    {Arquitecto de Software}
    {Ingrese como programador junior, desempeñando cargos como la creación de
        aplicaciones base para el desarrollo, tareas de diseño de interfaces
        como de lógica de negocio, integración de componentes, e investigación
        de tecnologías.

        Proyectos:
        \begin{itemize}
            \item Sistema Integral de Gestión Hospitalaria (Hospital de
                Clínicas)
            \item Sistema de Gestión Convergente de Nueva Generación (Copaco)
        \end{itemize}}
    {5280}
\entry
    {2011 - 2012}
    {Konecta S.A.}
    {Programador Junior}
    {Como programador junior, mis responsabilidades incluían el diseño y
        desarrollo de software para dispositivos móviles, y la creación de
        servicios webs, tanto servicios que interactuaban directamente con
        aplicaciones móviles, como otros que interactuaban con portales web
        \\}
    {1408}
\entry
    {2011 - 2011}
    {Global S.A.}
    {Programador Junior}
    {Como programador junior mis funciones iban desde correcciones y pruebas de
        aplicaciones en fase final de desarrollo, hasta realizar funciones
        relacionadas con el mantenimiento de la base de datos y su correcto
        funcionamiento. Además de administración de servidores. \\}
    {300}
\entry
    {2010 - 2011}
    {Bunker Games}
    {Programador}
    {Como programador principal y único, mis responsabilidades iban desde la
        correcto funcionamiento del videojuego, la integración del entorno
        gráfico, programación de la inteligencia artificial, cálculo de
        recorridos, optimización, etc. \\}
    {1408}

\end{entrylist}

\section{Freelancer}
\proyectof
    {2022 - act}
    {\href{https://github.com/InstIDEA/controlciudadano}{Portal Control Ciudadano}}
    {\href{https://www.controlciudadanopy.org/}{Link}}
    {Portal de control de gastos públicos}
    {Portal para exploración Datos Abiertos, desarrollado para
        realizar un control de los gastos del \textbf{COVID-19} y analizar el 
        crecimiento patrimonial de funcionarios públicos en base a sus 
        Declaraciones Juradas de Bienes y Renta.

        La recolección y limpieza de datos se realizo con Apache Airflow,
        adicionalmente se desarrollo un front end en React y un backend 
        en nodejs. 
        
        Para almacenar los datos se utilizo PostgreSQL y elasticsearch, como CI/CD 
        se utilizó Github Actions.
     }
    {300}
\proyectof
    {2018 - 2022}
    {\href{https://www.pemoi.com.py}{Pemoi}}
    {\href{https://www.pemoi.com.py/}{Yo Dono}}
    {Plataforma de Crowfounding para campañas políticas.}
    {Consiste en una aplicación multiplataforma~(Andorid e iOS) desarrollada utilizando
     Ionic, servicios de firebase (social login, realtime database, deep links, functions, 
     storage, etc), fabric.io (beta, crashlytics, fastlane, etc).
     
     Incluye un sistema de pagos digitales utilizando la plataformas de
     Tigo Money, Bancard S.A. y \href{https://www.pagopar.com.py}{Pagopar}.
     
     }
    {300}
\proyectof
    {2018 - act}
    {\href{http://github.com/avolpe/cotizacion}{Cotizaciones de monedas en el Paraguay}}
    {\href{https://cotizaciones.volpe.com.py}{Link}}
    {Herramienta para encontrar la mejor cotización para varias monedas,
    utiliza información de varias casas de cambio.}
    {Desarrollado con Vue y Java (SpringBoot), completamente OpenSource}
    {300}
\proyectof
    {2017 - act}
    {\href{http://github.com/avolpe/set-ruc-finder}{Indice de contribuyentes del Paraguay}}
    {\href{https://set.volpe.com.py}{Link}}
    {Herramienta para buscar RUCs válidos en el Paraguay, también provee algoritmos
    para validaciones y acceso a la base completa de contribuyentes del Paraguay}
    {Desarrollado con React y Rust, completamente OpenSource}
    {300}
\proyectof
    {2015}
    {\href{https://apps.facebook.com/pechugon_pepe}{PepeChugon Game}}
    {\href{http://www.animata.com.py/}{Animata}}
    {Consiste en un juego multiplataforma~(Android, Facebook Canvas y Windows)
        y un servidor donde se almacena la información y el rendimiento de los
        usuarios. El juego fue desarrollado para la campaña publicitaria del día
        del niño.}
    {El Juego fue desarrollado utilizando Unity3D y un componente de backend 
    desarrollado utilizando OpenShift con PostgreSQL.}
    {300}
        

\section{Experiencia académica}

\clase
    {2016}
    {Curso de Verano}
    {\href{http://www.pol.una.py/cursosverano/index.php?option=com_content&view=article&layout=edit&id=91}{Innovación
            social y transparencia mediante datos abiertos}} 
    {Auxiliar}
    {FP-UNA}
    {El curso busca crear una comunidad de personas que puedan aportar ideas
        innovativas con alto impacto social y que fomenten la transparencia en
        el uso de los recursos públicos. }
\clase
    {2016}
    {Licenciatura en ciencias Informáticas}
    {Algoritmos I y II} 
    {Auxiliar de enseñanza}
    {FP-UNA}
    {Introducción a algoritmos, conceptos básicos y programación con SLE2 y C.}

\clase
    {2016}
    {Ingeniería en informática}
    {Gestión de Centro de Cómputos} 
    {Auxiliar de enseñanza}
    {FP-UNA}
    {}

\newpage
\section{Educación}
\begin{entrylist}
  \educacion
    {2008 - 2015}
    {Ingeniero en Informática}
    {Facultad Politécnica - Universidad Nacional de Asunción}
    {\textbf{Tesis de grado:} Juegos serios como apoyo a la formación de profesionales: Una aplicación al área de enfermería}
  \educacion
    {2005 - 2007}
    {Bachiller Técnico en Informática}
    {Colegio de San José}
    {}
\end{entrylist}

\begin{absolutelynopagebreak}
\section{Referencias}
\begin{entrylist}
    \educacion{M.Sc.}{Martin Abente}{}{}
    %\educacion{M.Sc.}{Guillermo Gonzalez}{}{}
    \educacion{M.Sc.}{Maria Elena Garcia}{}{}
    \educacion{Ph.D.}{Juan Pane}{}{}
\end{entrylist}
\end{absolutelynopagebreak}

\begin{flushleft}
    \emph{\today{}}
\end{flushleft}
\begin{flushright}
    \emph{Arturo Volpe}
\end{flushright}



\end{document}
